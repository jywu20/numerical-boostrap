\documentclass[oneside]{fduthesis}

\fdusetup{
  style = {
    font = lm,
    % 西文字体(包括数学字体)
    % 允许选项:
    %   font = garamond|libertinus|lm|palatino|times
    %
    % 注意:
    %   1. LaTeX 默认风格是 lm
    %   2. Satoshi 的讲义风格是 palatino
    %
    cjk-font = windows,
    % 中文字体
    % 允许选项:
    %   cjk-font = adobe|fandol|founder|mac|sinotype|sourcehan|windows
    %
    % 注意:
    %   1. 中文字体设置高度依赖于系统。各系统建议方案:
    %        windows:cjk-font = windows
    %        mac:    cjk-font = mac
    %        linux:  cjk-font = fandol(默认值)
    %   2. 除 fandol 和 sourcehan 外,其余字体均为商用字体,请注意版权问题
    %   3. 但 fandol 字体缺字比较严重,而 sourcehan 没有配备楷体和仿宋体
    %   4. 某些字体会有 Font "xxx" does not contain requested Script "CJK" 的警告,可以忽略
    %
    font-size    = -4,
    bib-backend  = bibtex,
    bib-resource = {main.bib},
  },
  info = {
      title            = {xxx},
      % title            = {如果标题很长,可以用英文逗号分成两行},
      author           = {xxx},
      department       = {物理学系},
      major            = {物理学},
      supervisor       = {xxx},
      supervisor-title = {xxx},
      affiliation      = {物理学系},
      student-id       = {xxx},
      keywords         = {中文关键字, 关键字二},
      keywords*        = {English keywords, physics},
      % 中英文关键字均使用英文逗号分隔
  }
}

% 需要的宏包可以自行调用
\usepackage{booktabs}
\usepackage{physics}
\usepackage[linkcolor=black,menucolor=black]{hyperref}
\usepackage{prettyref}

\newrefformat{chap}{第\ref{#1}章}
\newrefformat{sec}{第\ref{#1}节}
\newrefformat{note}{注\ref{#1}}
\newrefformat{fig}{图\ref{#1}}
\newrefformat{alg}{算法\ref{#1}}

\newcommand{\hilbertH}{\symcal{H}}
\newcommand{\ee}{\symrm{e}}
\newcommand{\ii}{\symrm{i}}

% Disable unsupported commands in bookmark titles 
\pdfstringdefDisableCommands{%
  \def\\{}%
  \def\texttt#1{<#1>}%
  \def\mathbb#1{#1}%
}
\pdfstringdefDisableCommands{\def\eqref#1{(\ref{#1})}}

\makeatletter
\pdfstringdefDisableCommands{\let\HyPsd@CatcodeWarning\@gobble}
\makeatother

% 图片路径
% \graphicspath{{images/}}

% 目录深度,只保留到 \section
\setcounter{tocdepth}{2}

\begin{document}

% 前置部分包含目录、中英文摘要以及符号表等
\frontmatter

% 目录
\tableofcontents
% 插图目录
%\listoffigures
% 表格目录
% \listoftables

\begin{abstract}
摘要摘要
\end{abstract}

\begin{abstract*}
If using Overleaf, you can upload all the files and use \verb|xelatex| to compile.
\end{abstract*}

% 主体部分是论文的核心
\mainmatter

\ctexset{chapter/pagestyle=fancy}

% 建议采用多文件编译的方式
% 比较好的做法是把每一章放进一个单独的 tex 文件里,并在这里用 \include 导入,例如
%   \include{chapter1}
%   \include{chapter2}
%   \include{chapter3}

\chapter{绪论}

\section{引言}

\begin{figure}[h]
  \centering
  \includegraphics{example-image.pdf}
  \caption{模型示意图}
  \label{fig:example}
\end{figure}

\section{Bootstrap技术示例:共性Bootstrap}

\section{数值Bootstrap的形式理论}

\chapter{一维非线性谐振子的数值bootstrap}

\section{一维非线性谐振子——一个$0+1$维非微扰场论}

\section{非线性SDP优化}

\subsection{$\expval{x^n}$的非线性递推关系}

\subsection{可行域,基态和第一激发态}

\subsection{非线性SDP在大规模问题上的不可行性}

\section{线性SDP优化}

\subsection{对易子和等式关系的符号计算}

\subsection{构建优化问题}

\subsection{收敛性问题}

\begin{figure}[h]
  \centering
  \caption{试验合金的马氏体相变温度 Ms 及反铁磁转变温度 TN}
  \begin{tabular}{ccccc}
    \toprule
      & 1\# & 2\# & 3\# & 4\# \\
    \midrule
      Mn (at\%) & 86.4 & 80.8 & 71.3  & 61.4    \\
      TN / °C   & 173  & 157  & 128   & 208     \\
      Ms / °C   & 180  & 20   & $-40$ & $< -60$ \\
    \bottomrule
  \end{tabular}
  \label{tab:example}
\end{figure}

\chapter{Hubbard模型的数值bootstrap}

\section{}

\chapter{总结与展望}


引用~\cite{feynman2011feynman},行内引用~\parencite{feynman2011feynman}。

% 后置部分包含参考文献、声明页(自动生成)等
\backmatter

% 打印参考文献列表
\printbibliography

\chapter{致\quad 谢}

\chapter{附\quad 录}

\end{document}
