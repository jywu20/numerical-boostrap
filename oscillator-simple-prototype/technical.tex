\documentclass{beamer}
\usepackage{physics}
\usepackage{amsmath}
\usepackage{tikz}
\usepackage{mathdots}
\usepackage{yhmath}
\usepackage{cancel}
\usepackage{color}
\usepackage{siunitx}
\usepackage{array}
\usepackage{multirow}
\usepackage{amssymb}
\usepackage{textcomp, gensymb}
\usepackage{tabularx}
\usepackage{extarrows}
\usepackage{booktabs}
\usetikzlibrary{fadings}
\usetikzlibrary{patterns}
\usetikzlibrary{shadows.blur}
\usetikzlibrary{shapes}
\usepackage{listings}
\usepackage{hyperref}

%Information to be included in the title page:
\title{Technical details in linear SDP bootstrap of $x^4$ anharmonic oscillator}
\author{Jinyuan Wu}
\institute{Department of Physics, Fudan University}
\date{2021}

\usetheme{Madrid}

\newcommand{\concept}[1]{\textbf{#1}}
\newcommand{\ee}{\mathrm{e}}
\newcommand{\ii}{\mathrm{i}}

\begin{document}

\frame{\titlepage}

\begin{frame}
\frametitle{Basis of operator space}

\textbf{The operator space basis}

\begin{itemize}
    \item All in the form of $x^m p^n$ 
    \item Cutoff: $0 \leq m, n \leq 2L$
    \item In the $M_{ij} = \expval*{O_i O_j}$ matrix: this means the cutoff is $0 \leq i, j \leq L$
\end{itemize}    

\textbf{Indexing of each operator}

\begin{itemize}
    \item Two digit base-$2L$ number $\overline{mn}$
    \item Min = $0$, Max = $(2 L + 1)^2 - 1$ 
    \item Analogy: base-10, $2L = 9$, Min = 0, Max = 99 = $(9 + 1)^2 - 1$
\end{itemize}

\textbf{Implementation}

\begin{itemize}
    \item \texttt{xpopstr\_index(x\_power, p\_power)}: from $(m, n)$ to $\overline{mn}$
    \item \texttt{index\_to\_xpower(idx)}: from $\overline{mn}$ to $m$
    \item \texttt{index\_to\_ppower(idx)}: from $\overline{mn}$ to $n$
\end{itemize}

\end{frame}

\begin{frame}
\frametitle{Operator algebra}

\textbf{Operator algebra when formulating the problem}

\begin{itemize}
    \item $\comm*{H}{O}$: normal ordered commutator between $x^{m_1} p^{n_1}$ and $x^{m_2} p^{n_2}$
    \begin{itemize}
        \item Equivalent to commutator between $x^m$ and $p^n$
    \end{itemize}
    \item $O_i O_j$: normal ordered multiplication of $x^{m_1} p^{n_1}$ and $x^{m_2} p^{n_2}$
    \begin{itemize}
        \item Equivalent to normal ordering of $p^{n} x^m$
        \item Equivalent to commutator between $x^m$ and $p^n$
    \end{itemize}
    \item Normal ordered commutator between $x^m$ and $p^n$: McCoy's formula\footnote{See \url{https://en.wikipedia.org/wiki/Canonical_commutation_relation}}
\end{itemize}

\textbf{Fundamental operations we need}

\begin{itemize}
    \item $x^l \cdot x^m p^n \cdot p^k$ -- implemented as \texttt{xpopstr\_left\_x\_right\_p\_mul}
    \item $\comm*{x^m}{p^n}$ -- implemented as \texttt{xpopstr\_comm}
\end{itemize}

\end{frame}

\begin{frame}
\frametitle{Constructing $M$ matrix}

    

\end{frame}

\begin{frame}
\frametitle{Imaginary number in linear SDP}

Since CSDP doesn't support complex number in the semidefinite constraint, we need to 
replace $1$ by $I_{2 \times 2}$ and $\ii$ by $\pmqty{0 & -1 \\ 1 & 0}$.

\end{frame}

\end{document}